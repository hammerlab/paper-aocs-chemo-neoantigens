\subsection*{Background}

Several chemotherapies including platinum compounds~\cite{Hannan_1989}, cyclophosphamide~\cite{Anderson_1995}, and etoposide~\cite{NAKANOMYO_1986} have long been known to be mutagenic, and recent studies have found evidence for chemotherapy-induced mutations in post-treatment acute myeloid leukaemia~\cite{Ding_2012}, glioma~\cite{Johnson_2013}, and esophageal adenocarcinoma~\cite{Murugaesu_2015}. At the same time, there is increasing appreciation for the importance of T cell neoantigens arising from somatic mutations in anti-tumor immune responses~\cite{Schumacher_2015}. Patients with more neoantigens have better prognosis~\cite{Brown_2014} and are more likely to respond to checkpoint blockade immunotherapy~\cite{Van_Allen_2015,Rizvi_2015}. It has been suggested that a relatively low neoantigen burden in ovarian cancer may explain the limited effectiveness of certain immunotherapies in this disease~\cite{Martin_2016}, raising the question if chemotherapy might be used specifically to increase neoantigen burden. However, the effect of standard chemotherapy regimes on tumor mutations and resulting neoantigens in both the neoadjuvant and adjuvant settings is unknown.

The Australian Ovarian Cancer Study (AOCS) performed whole genome sequencing on 115 cancer samples from 93 donors with high grade serous ovarian carcinoma~\cite{Patch_2015}. RNA-sequencing was additionally performed on 98 of the samples. Eighty samples were taken before chemotherapy, 30 after adjuvant chemotherapy, and 5 after neoadjuvant therapy. Treatment regimes varied but always included combination carboplatin/paclitaxel. In their primary analysis, Patch et al. reported that post-treatment samples harbored more somatic mutations than pre-treatment samples and exhibited evidence of chemotherapy-associated mutations. Here we extend these results by first quantifying the overall change in mutations and resulting neoantigens with treatment and, secondly, assessing how much of this change is attributable to chemotherapy-associated mutational signatures.

% Given these results, we wondered if this dataset could be used to assess the effect of chemotherapy on mutation and neoantigen burden.
% analysis might be extended to quantify the number of neo whether chemotherapy leads to more neoantigens.

As paclitaxel/carboplatin therapy following surgery is standard of care, all AOCS relapse samples were exposed to chemotherapy. Therefore, the effect of chemotherapy on mutational burden cannot be statistically separated from those of drift and relapse in this cohort. Mutational signature deconvolution provides an alternative. In this method, the SNVs observed in a sample are counted by trinucleotide context, and a combination of signatures, each corresponding to a mutagenic process, is found that best explains the observed counts~\cite{Alexandrov2013}. Mutational signatures may be discovered \textit{de novo} from large cancer sequencing projects, but for smaller studies it is preferable to deconvolve into known signatures~\cite{Rosenthal_2016}, as we do here.

The COSMIC Signature Resource curates 30 signatures from pan-cancer analyses; a number of these have known associations with mutagenic processes such as disruption of mismatch repair pathways or ultraviolet light exposure~\cite{364242}. While signatures for chemotherapy exposure have not been established from human studies, two recent reports provide data on mutations detected in chemotherapy-exposed \textit{C. Elegans} organisms~\cite{Meier_2014} and a \textit{G. Gallus} (chicken) cell line~\cite{Szikriszt_2016}. The \textit{C. Elegans} study considered cisplatin and other compounds across a range of DNA repair-deficient knockout models. The \textit{G. Gallus} study looked at a number of chemotherapies, including cisplatin, cyclophosphamide, and etoposide, in wildtype chicken cell lines. Using the mutations identified in these studies, we extracted signatures for cisplatin, cyclophosphamide, and etoposide, and included 

