\section*{Results}

\begin{table}[htbp]


\begin{tabular}{lllll}
\toprule
{} & samples (RNA) &         Mutations &   Neoantigens & Expressed Neoantigens \\
\midrule
ascites pre-treatment                 &         4 (4) &  10336 $\pm$ 1000 &  202 $\pm$ 60 &           78 $\pm$ 30 \\
ascites post-treatment                &       24 (20) &  13757 $\pm$ 1000 &  300 $\pm$ 50 &          145 $\pm$ 30 \\
\textit{model adjusted change (\%)} &               &       57 $\pm$ 67 &  72 $\pm$ 107 &         125 $\pm$ 149 \\
\hline
solid pre-treatment                   &       75 (69) &    7902 $\pm$ 900 &  155 $\pm$ 20 &            65 $\pm$ 9 \\
solid post-treatment                  &        12 (5) &  11250 $\pm$ 3000 &  253 $\pm$ 80 &           38 $\pm$ 20 \\
\textit{model adjusted change (\%)}   &               &        7 $\pm$ 28 &    5 $\pm$ 39 &          -41 $\pm$ 32 \\
\bottomrule
\end{tabular}



\caption{Cohort size and means.}
\label{tab:cohort}
\end{table}

\begin{figure}[htbp]
\centering
\includegraphics[scale=1.0]{figures/signatures.pdf}
\caption{Mutational signature deconvolution. }
\label{fig:signatures}
\end{figure}

\subsection*{Change in mutations and neoantigens at recurrence}
The treated samples showed more mutations, neoantigens, and, in the case of ascites samples, more expressed neoantigens (Table \ref{tab:cohort}).

Total somatic mutation burden increased at recurrence in 11 of the 12 donors with pre- and post-treatment samples. A Bayesian model integrating both paired and unpaired samples found the post-treatment timepoint to be associated with a 52\% (95\% credible region -1--134) increase in somatic mutations (Figure \ref{bayesian}), with a 93\% posterior probability that post-treatment timepoint was associated with at least a 5\% increase in mutations. 

We identified 18,491 potential neoantigens, defined as mutated peptides predicted to bind autologous MHC class I with affinity $\leq 500$nm (Table \ref{tab:cohort}). All but 26 (0.14\%) neoantigens were private to a single donor. The number of neoantigens tracked the increase in mutational burden at relapse. In the Bayesian analysis, treated samples had 59\% (-15--199) more neoantigens, and, for ascites samples, 106\% (7--305) more expressed neoantigens. Interestingly, solid tumor samples showed an increase in neoantigens but a 44\% (4--67) decrease in expressed neoantigens.

Composition of neoantigens: SNV, MNVS, indels pre and post


\subsection*{Mutation signatures}
Signature deconvolution using the 30 mutational signatures curated by COSMIC\cite{364242}, plus four additional signatures extracted from a study of cisplatin-exposed \textit{C. Elegans}\cite{Meier_2014}, and a chicken cell line exposed to cisplatin, cyclophosphamide, and etoposide\cite{Szikriszt_2016} found that the dominant signatures were largely the same in pre- and post-treatment samples (Supplementary Figure \ref{sfig:supp_signatures}). In both groups, the top signatures were \textit{Signature 3}, associated with BRCA disruption and accounting for 37\% (bootstrap 95\% CI 34-39) of mutations, \textit{Signature 8}, of unknown etiology and accounting for 19\% (17-20) of mutations, and \textit{Signature 1}, associated with age at diagnosis and accounting for 10\% (8-11) of mutations. The remaining mutations were attributed to different signatures in various samples. No samples showed presence of either cisplatin signature or the etoposide signature. Two pre-treatment and five post-treatment samples had evidence for the cyclophosphamide signature, but none of these patients had a clinical record of cyclophosphamide use, and 0/10 samples with documented cyclophosphamide use exhibited the signature.

For better sensitivity to detect signatures operative at relapse, we next focused on the 14 samples from 12 donors with paired pre- and post-treatment samples. For each donor, we extracted the mutations that had evidence only in the treated samples using a stringent read filter requiring at least 30 reads coverage and zero variant reads in the pre-treatment samples. Of 229,132 SNV mutations in the relapse samples for these donors, we identified 106,171 such ``unique to treated'' mutations, over which we performed signature deconvolution. Two samples (AOCS-092-13, AOCS-095-13) showed evidence for the chicken cisplatin signature. All four donors (AOCS-064, AOCS-093, AOCS-137, AOCS-139) with a clinical record of cyclophophamide treatment showed evidence for the cyclophosphamide signature. However, five donors (AOCS-086, AOCS-088, AOCS-091, AOCS-092, AOCS-095) without a record of cyclophosphamide use also exhibited this signature. \textit{Signature 1} was completely absent, consistent with its association with slow ongoing mutagenic processes present in all tissue. All samples showed \textit{Signature 3} (BRCA disruption), and 9 of 14 samples showed \textit{Signature 8} (unknown etiology), indicating that the processes likely play a role in the continued evolution post-treatment. Other signatures appeared only sporadically.

sensitivity analysis to detect cisplatin signature


