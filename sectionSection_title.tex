\section*{Results}
\subsection*{Somatic mutation and neoantigen burden}

We first considered the total somatic mutation burden of the samples. For 15/16 donors, the median number of detectable somatic mutations increased at relapse \ref{fig:total_mutations}, a statistically significant effect (p \lt 0.001). The change ranged from -17\% to 134\% with a mean of 27\%.  

The number of expressed somatic mutations predicted to generate MHC I peptides with affinity 500nm or stronger for   9/13 donors 

As expected somatic mutation and neoantigen burden were highly correlated with no substantial deviations

Therapy vs. burden


\subsection*{Mutational signatures}
To explore possible etiologies for the increased mutational burden at recurrence, we deconvolved the mutations present in the primaries as well as the unique-to-recurrence mutations into known signatures. We used the 30 signatures curated by COSMIC (http://cancer.sanger.ac.uk/cosmic/signatures) plus an additional signature extracted from a study of C. Elegans after cisplatin exposure [Meier et al]. Consistent with results originally reported by Patch et al on this data, the mutations present in the primary tumors were attributable mostly to Signatures 1 and 3, which are associated with Age and BRCA disruption, respectively. We deconvolved the mutations unique to the recurrence samples. 

We note that the C(C>T)C mutations found at higher rates in the relapse samples do not correspond to the signature found in \textit{C. Elegans} (supplementary figure).

\subsection*{Allelic fractions}



