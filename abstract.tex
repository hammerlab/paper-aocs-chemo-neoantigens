Patients with highly mutated tumors have better prognosis in a variety of clinical contexts. Part of this effect may be due to an increased number of T-cell neoantigens, resulting in more immunologically foreign tumors. Many chemotherapies including platinum compounds are known to be mutagenic, but the contribution of standard treatment protocols to detectable mutational and neoantigen burden in most human cancers is unknown. Using 115 samples from the Australian Ovarian Cancer Study, we quantify the effect of chemotherapy treatment on mutation and neoantigen burden in ovarian cancer. We find that, while chemotherapy-treated samples have an estimated 57\% more mutations and 65\% more potential neoantigens than primary samples, less than 15\% of mutations and neoantigens match signatures derived from studies of chemotherapy-exposed \textit{C. Elegans} organisms and \textit{G. Gallus} cell lines. In both treated and untreated samples, the majority of mutations and neoantigens are instead attributed to signatures associated with BRCA pathway disruption (COSMIC Mutational Signature 3), a process of unknown etiology (Signature 8), and spontaneous deamination of 5-methylcytosine (Signature 1).

% These results suggest that standard chemotherapy regimes are not a dominant source of neoantigens in this disease.

% These results demonstrate an approach for quantifying chemotherapy-associated mutations and neoantigens and suggest that standard chemotherapy regimes are not a dominant source of neoantigens in ovarian cancer.