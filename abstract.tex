Patients with highly mutated tumors have better prognosis in a variety of clinical contexts. Part of this effect may be due to an increased number of T-cell neoantigens, resulting in more immunologically foreign tumors. Several mainstay chemotherapies including platinum compounds are known to be mutagenic, but their contribution to detectable mutational and neoantigen burden in treated human cancers is poorly understood. Combining 115 samples from the Australian Ovarian Cancer Study with two studies of chemotherapy-exposed animals, we quantify the effect of chemotherapy treatment on mutation and neoantigen burden. We find that, while chemotherapy-treated samples have an estimated 57\% more mutations and 65\% more potential neoantigens than primary samples, less than 15\% of mutations and neoantigens match experimentally-derived signatures of chemotherapy exposure. Signatures associated with BRCA pathway disruption (COSMIC Mutational Signature 3), a process of unknown etiology (Signature 8), and spontaneous deamination of 5-methylcytosine (Signature 1) instead account for the majority of mutations and neoantigens in both treated and untreated samples. These results suggest that standard chemotherapy regimes are not a dominant source of neoantigens in ovarian cancer.

% These results demonstrate an approach for quantifying chemotherapy-associated mutations and neoantigens and suggest that standard chemotherapy regimes are not a dominant source of neoantigens in ovarian cancer.