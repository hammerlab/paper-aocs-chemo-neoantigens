Cancer patients with highly mutated tumors have better prognosis in a variety of clinical contexts. Part of this effect may be due to an increased number of T-cell neoantigens, resulting in more immunologically foreign tumors. Several mainstay chemotherapies including platinum compounds are known to be mutagenic, but their contribution to detectable mutational and neoantigen burden in treated human cancers is unknown. Based on an analysis of 115 whole genome sequenced samples from the Australian Ovarian Cancer Study and two studies of chemotherapy-exposed organisms, here we show that, while relapsed samples have substantially more mutations and potential neoantigens than primary samples, less than 15\% match experimentally-derived mutational signatures of chemotherapy exposure. Signatures associated with BRCA pathway disruption (COSMIC Mutational Signature 3), a process of unknown etiology (Signature 8), and spontaneous deamination of 5-methylcytosine (Signature 1) instead account for the majority of mutations and neoantigens in both treated and untreated samples. These results suggest that standard chemotherapy regimes are not a dominant source of neoantigens in ovarian cancer.

% These results demonstrate an approach for quantifying chemotherapy-associated mutations and neoantigens and suggest that standard chemotherapy regimes are not a dominant source of neoantigens in ovarian cancer.