\section*{Methods}

\subsection*{Clinical sample information}
We grouped the AOCS samples into three sets --- ``primary/untreated,'' ``primary/treated,'' and ``relapse/treated'' --- according to collection time point and chemotherapy exposure. The primary/untreated group consists of 75 primary debulking surgical samples and 4 samples of drained ascites. The primary/treated group consists of 5 primary debulking surgical samples obtained from patients pretreated with chemotherapy prior to surgery (neoadjuvant chemotherapy). The relapse/treated group consists of 24 relapse or recurrence ascites samples, 5 metastatic samples obtained in autopsies of two patients, and 1 solid tissue relapse surgical sample, all of which were obtained after prior exposure to one or more lines of chemotherapy.  In summary, these groupings yield 79 primary/untreated samples, 5 primary/treated samples, and 30 relapse/treated samples. Sample and clinical information including chemotherapy treatments is listed in Additional File 1.

Independent of treatment, ascites samples trend toward more detected mutations, perhaps due to increased intermixing of clones. We therefore stratified by tissue type (solid tumor or ascites) when comparing the mutation and neoantigen burdens of pre- and post-treatment samples.

\subsection*{Mutation calls}
We analyzed the mutation calls published by Patch et al.~\cite{Patch_2015} (Additional File 2). DNA and RNA sequencing reads were downloaded from the European Genome-phenome Archive under accession EGAD00001000877. Adjacent SNVs from the same patient were combined to form multinucleotide variants (MNVs). 

We considered a mutation to be present in a sample if it was called for the patient and more than 5 percent of the overlapping reads (and at least 6 reads total) supported the alternate allele. We considered a mutation to be expressed if there were 3 or more RNA reads supporting the alternate allele. In the analysis of paired pre- and post-treatment samples from the same donor, we defined a mutation as unique to the post-treatment sample if the pre-treatment sample contained greater than 30 reads coverage and no variant reads at the site.

\subsection*{HLA typing, variant annotation, and MHC binding prediction}
\begin{sloppypar}
HLA typing was performed using a consensus of seq2HLA~\cite{Boegel_2012} and OptiType~\cite{Szolek_2014} across the samples for each patient (Additional File 3). The most disruptive effect (in terms of amino acid sequence) of each protein-changing variant was predicted using Varcode (http://github.com/hammerlab/varcode). For insertions or deletions (indels) that were predicted to disrupt the reading frame, all downstream peptides potentially generated up to a stop codon were considered. Class I MHC binding predictions were performed for peptides of length 8--11 using NetMHCpan 2.8~\cite{Lundegaard_2008} with default arguments (predicted neoantigens are listed in Additional File 2).
\end{sloppypar}

\subsection*{Mutational signatures}
The use of mutational signatures is necessary because it is not possible to distinguish chemotherapy-induced mutations from temporal effects when comparing primary and relapse samples by mutation count alone. A mutational signature ascribes a probability to each of the 96 possible single-nucleotide variants, where a variant is defined by its reference base pair, alternate base pair, and base pairs immediately adjacent to the mutation. Signatures have been associated with exposure to particular mutagens, age related DNA changes, and disruption of DNA damage repair pathways due to somatic mutations or germline risk variants in melanoma, breast, lung and other cancers~\cite{Alexandrov2013}, and provide a means of identifying the contribution that chemotherapy may make to the mutations seen in post-treatment samples. For example, the chemotherapy temozolomide has been shown to induce mutations consisting predominantly of $C \rightarrow T$ (equivalently, $G \rightarrow A$) transitions at CpC and CpT dinucleotides~\cite{Johnson_2013}. To perform deconvolution, the single nucleotide variants (SNVs) observed in a sample are tabulated by trinucleotide context, and a combination of signatures, each corresponding to a mutagenic process, is found that best explains the observed counts. Mutational signatures may be discovered \textit{de novo} from large cancer sequencing projects but for smaller studies it is preferable to deconvolve using known signatures~\cite{Rosenthal_2016}.

The Catalogue Of Somatic Mutations In Cancer (COSMIC) Signature Resource curates 30 signatures discovered in a pan-cancer analysis of untreated primary tissue samples. While signatures for exposure to the chemotherapies used in ovarian cancer have not been established from human studies, two recent reports provide data on mutations detected in cisplatin-exposed \textit{C. Elegans}~\cite{Meier_2014} and a \textit{G. Gallus} cell line exposed to several chemotherapies including cisplatin, chyclophosphamide, and etoposide ~\cite{Szikriszt_2016}. From the SNVs identified in these studies, we defined two signatures for cisplatin, a signature for cyclophosphamide, and a signature for etoposide (Figures~\ref{fig:supp_extracted_signatures_chicken} and~\ref{fig:supp_extracted_signatures_worm}). As both studies sequenced replicates of chemotherapy-treated and untreated (control) samples, identifying a mutational signature associated with treatment required splitting the mutations observed in the treated group into background and treatment effects. We did this using a Bayesian model for each study and chemotherapy drug separately.

Let $C_{i,j}$ be the number of mutations observed in experiment $i$ for mutational trinucletoide context $0 \leq j < 96$. Let $t_i \in \{0,1\}$ be 1 if the treatment was administered in experiment $i$ and 0 if it was a control. We estimate the number of mutations in each context arising due to background (non-treatment) processes $B_j$ and the number due to treatment $T_j$ according to the model:

\[
C_{i,j} \sim \mathit{Poisson}(B_j + t_i T_j)
\]

We fit this model using Stan~\cite{Gelman_2015} with a uniform (improper) prior on the entries of $B$ and $T$. The treatment-associated mutational signature $N$ was calculated from a point estimate of $T$ as:

\[
N_j = \left ( \frac{T_j}{\sum_{j'}{T_{j'}}} \right ) \left ( \frac{h_j}{m_j} \right )
\]

where $h_j$ and $m_j$ are the number of times the reference trinucleotide $j$ occurs in the human and preclinical model (\textit{C. Elegans} or \textit{G. Gallus}) genomes, respectively.

%The signature deconvolution was performed with the deconstructSigs\cite{Rosenthal_2016} package using the following parameters passed to \texttt{whichSignatures()}:

Signature deconvolution was performed with the deconstructSigs\cite{Rosenthal_2016} package using the 30 mutational signatures curated by COSMIC~\cite{364242} extended to include the putative chemotherapy-associated signatures (Additional Files 4 and 5). When establishing whether a signature was detected in a sample, we applied the 6\% cutoff recommended by the authors of the deconstructSigs package. Signatures assigned weights less than this threshold in a sample were considered undetected.

%\texttt{contexts.needed=TRUE}, \texttt{signature.cutoff=0.0}, \texttt{tri.counts.method="default"}

To estimate the number of SNVs and neoantigens generated by a signature, for each mutation in the sample we calculated the posterior probability that the signature generated the mutation, as described below. The sum of these probabilities gives the expected number of SNVs attributable to each signature. For neoantigens, we weighted the terms of this sum by the number of neoantigens generated by each mutation.

Suppose a mutation occurs in context $j$ and sample $i$. We calculate $\Pr[s \mid j]$, the probability that signature $s$ gave rise to this mutation, using Bayes' rule:

\[
\Pr[s \mid j] = \frac{\Pr[j \mid s] \Pr[s]}{\sum_{s'}{\Pr[j \mid s']\Pr[s']}} = \frac{H_{s,j} \, D_{i,s}}{\sum_{s'}{H_{s',j} \, D_{i,s'}}}
\]

where $D_{i,s}$ gives the contribution of signature $s$ to sample $i$ and $H_{s,j}$ is the weight for signature $s$ on mutational context $j$. For treated samples with a pre-treatment sample available from the same patient, we deconvolved signatures for both the full set of mutations and for the mutations detected only after treatment. When calculating $\Pr[s \mid j]$ for these samples, for each mutation we selected the appropriate deconvolution matrix $D_{i,s}$ based on whether the mutation was unique to the post-treatment sample.

% Previous studies have often performed signature extraction \textit{de novo} and interpreted the results by comparing to existing signatures. We deconvolved onto existing signatures, as enables a more straightforward interpretation of this relatively small data set. Deconvolution was performed with the deconstructSigs~\cite{Rosenthal_2016} package for each sample individually. We used the 30 mutational signatures curated by COSMIC~\cite{364242} extended to include the putative chemotherapy-associated signatures. When establishing whether a signature was detected in a sample, we applied the 6\% cutoff recommended by the authors of the deconstructSigs package. Signatures assigned weights less than this threshold were considered undetected.
