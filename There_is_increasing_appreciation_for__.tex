There is increasing appreciation for the importance of T cell neoantigens arising from somatic mutations in anti-tumor immune responses~\cite{Schumacher_2015}. Patients with more neoantigens have better prognosis~\cite{Brown_2014} and are more likely to respond to checkpoint blockade immunotherapy\cite{Van_Allen_2015,Rizvi_2015}. Several chemotherapies including platinum compounds have long been known to be mutagenic\cite{Hannan_1989} and recent studies have found evidence for chemotherapy-induced mutations in human cancers\cite{Murugaesu_2015,Johnson_2013}. While there has been much study of how chemotherapy may otherwise enhance~\cite{Hato_2012,Machiels2001,Hodge2013} or impair~\cite{Litterman_2013} an anti-tumor immune response, the extent to which it induces tumor neoantigens has not been well assessed.

The Australian Ovarian Cancer Study (AOCS) performed whole genome sequencing on 115 cancer samples from 93 donors with high grade serous ovarian carcinoma\cite{Patch_2015}. Eighty samples were taken before chemotherapy and 35 after combination carboplatin/paclitaxel and other chemotherapy treatments. The primary AOCS analysis found that post-treatment samples harbored more somatic mutations than pre-treatment samples and exhibited evidence of chemotherapy-associated mutations. Here we extend these results by quantifying the change in mutational and neoantigen burden with treatment and estimating the number of neoantigens attributable to chemotherapy.

As adjuvant paclitaxel/carboplatin therapy is standard of care, all relapse samples in the AOCS cohort were taken after chemotherapy treatment. Therefore, the effect of chemotherapy on mutational burden cannot be statistically separated from those of drift and relapse in this dataset. This motivates an approach based on mutational signatures, a method for representing the single nucleotide variants (SNVs) observed in a sample as a weighted sum of several \textit{signatures} based on the reference, alternate, and immediately adjacent base-pairs of each SNV~\cite{Alexandrov2013}. Different mutagenic processes have different preferences for the mutations they induce, and thus different characteristic signatures. Thirty signatures are curated by COSMIC~\cite{364242} from large pan-cancer analyses, and a number of these have known associations with mutagenic processes such as BRCA disruption or ultraviolet light exposure. While signatures for chemotherapy exposure have not been established from human studies, there are two recent reports providing mutations detected in chemotherapy-exposed \textit{C. Elegans} organisms\cite{Meier_2014} and a \textit{G. Gallus} (chicken) cell line \cite{Szikriszt_2016}. The \textit{C. Elegans} study considered cisplatin and other compounds across a range of DNA repair-deficient knockout models. The \textit{G. Gallus} study looked at a number of chemotherapies, including cisplatin, cyclophosphamide, and etoposide, in wildtype chicken cell lines. Using the mutations identified in these studies, we extracted several putative signatures for cisplatin, cyclophosphamide, and etoposide, tested the extent to which these are consistent with clinical record, and quantified the number mutations observed in the AOCS cohort attributable to these signatures.
